\documentclass[12pt,a4paper]{article}
\usepackage[utf8x]{inputenc}
\usepackage{cite}
\usepackage{hyperref}
%\usepackage{float} %for H positioning


\title{Software requirements for caTools}
%\author{Sašo Skube, PSI}
\date{}

\renewcommand*{\theenumi}{\thesection.\arabic{enumi}}
\renewcommand*{\theenumii}{\theenumi.\arabic{enumii}}

\begin{document}

\maketitle

%\tableofcontents
%\newpage

%\section{Introduction}

\section{Requirements}
\begin{enumerate}
	\item The following tools shall be available:
	\begin{description}
		\item [caget] Reads and format value from channels.
		\item [cagets] Writes 1 to PROC field of a channel and reads it back after channel has finished processing.
		\item [caput] Writes a value to a channel.
		\item [caputq] Writes a value to a channel, but does not wait for the processing to finish.
		\item [cainfo] Displays detailed information about a channel.
		\item [camon] Monitors a set of channels and outputs their values on each change.
		\item [cado] Writes 1 to a channel, but does not wait for the processing to finish.
		\item [cawait] Monitors a channel and waits until specified conditions for the channel match. Then the value is displayed.
	\end{description}
	\item caTools shall be written in C programming language.
	\item caTools shall compile on the following platforms:
	\begin{itemize}
		\item SL6-x86 SL6-x86\_64,
		\item eldk52-e500v2,
		\item eldk42-ppc4xxFP.
	\end{itemize}
	\item Input and expected output shall be backwards compatible with existing PSI tools, except for \textbf{caInfo} tool and array handling.
	\item \textbf{caInfo} tool shall display all the information available in \texttt{DBR\_CTRL\_XXX} compound data type.
	\item The user shall be able to request specified compound data type (eg. \texttt{DBR\_TIME\_CHAR}).
	\item The user shall be able to select channel access timeout (eg. how long to wait for channel to connect).
	\item Date and time of the record execution shall be available (CA server date/time).
	\item Date and time of the record printout shall be available (client local data/time).
	\item \textbf{camon} shall have additional incremental time-stamps options:
	\begin{itemize}
		\item Time elapsed since start of program.
		\item Time elapsed since last update.
		\item Time elapsed since last update, by channel.
	\end{itemize}
	\item \textbf{camon} shall have an option to exit monitoring after user specified number of updates.
	\item \textbf{cawait} shall have an option to exit after a user specified timeout.
	\item The format of the \texttt{decimal} values shall be settable:
	\begin{itemize}
		\item Round the value to closest integer, round to next integer, round to previous integer.
		\item Scientific notation (mantissa/exponent) with specified precision - \%e format.
		\item Decimal floating point notation with specified precision - \%f format.
		\item Use the shortest representation between \%e and \%f.
		\item Override the PREC field defined in the record.
	\end{itemize}
	\item The user shall be able to set the format of the \texttt{integer} values to hex, binary or octal.
	\item The user shall be able to force interpretation of \texttt{enum} and \texttt{char} values as integers or characters / array of characters.
	\item Arrays shall be printed as a list of values, where the formatting shall be based on the value type (eg. char type is displayed as an ASCII character, double type as a decimal number).
	\item Array values separator shall be configurable.
	\item The user shall be able to set the requested amount of array elements to read or write.
%	\item The user shall be able to display the number of elements in an array (NORD record field) before printing out array values. 
	\item Severity and status shall be displayed when not in \texttt{NO\_ALARM} state.
	\item The user shall be able for force the severity and status to always be displayed.
	\item The user shall be able for force the severity and status to never be displayed.
	\item The user shall be able to hide channel name from the output.
	\item The user shall be able to hide engineering units (EGU) from displaying next to values.
	\item The default display output format will be in the following form: 
	\begin{verbatim}
		[date] [time] [channel name] [value] [EGU] [severity and status]
	\end{verbatim}
\end{enumerate}

%\bibliographystyle{plain}
%\bibliography{references}

\end{document}
