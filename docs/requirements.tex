\documentclass[12pt,a4paper]{article}
\usepackage[utf8x]{inputenc}
\usepackage{cite}
\usepackage{hyperref}
%\usepackage{float} %for H positioning


\title{Software requirements for caTools}
%\author{Sašo Skube, PSI}
\date{}

\renewcommand*{\theenumi}{\thesubsection.\arabic{enumi}}
%\renewcommand*{\theenumii}{\theenumi.\arabic{enumii}}

\begin{document}

\maketitle

%\tableofcontents
%\newpage

\section{Introduction}
caTools represent a set of tools used to read, write and display information about EPICS channels. Two sets of tools currently exist. First set is written in TCL programming language and have been in use at PSI for many years. Later a second, set of tools was added to EPICS base. The basic functionalities of both are equal, but each has different additional features. Furthermore, the input and output of both tools is not compatible. 

The new version of the TCL programming language is not compatible to the previous ones, so the existing PSI tools cannot be used with the new TCL version. Thus a new set of caTools is needed, that will be backwards compatible with the existing PSI tools and that will have additional features from EPICS base tool set added.

An analysis of the current tools is available inside a Jira ticket~\cite{jira_analyze}. A preliminary list of command line arguments for the new tools is available inside a Jira ticket~\cite{jira_requirements} as \texttt{arguments.pdf} attachment. The rest of this document describes the requirements for the new set of caTools.

\section{Requirements}
\subsection{General}
\begin{enumerate}
	\item The default output format will be in the following form: 
	\begin{verbatim}
		[date] [time] [channel name] [value] [EGU] [severity and status]
	\end{verbatim}
	\item The following tools shall be available:
	\begin{description}
		\item [caget] Reads and formats a value from each channel specified by the user. This tool shall honor the default output format.
		\item [cagets] Writes 1 to PROC field of a channel and reads it back after channel has finished processing. This tool shall honor the default output format.
		\item [caput] Writes a value to a channel and reads it back. This tool shall honor the default output format.
		\item [caputq] Writes a value to a channel, but does not wait for the processing to finish. Does not have any output (except if an error occurs).
		\item [cainfo] Displays detailed information about a channel. Has a custom output.
		\item [camon] Monitors a set of channels and outputs their values on each change. This tool shall honor the default output format.
		\item [cado] Writes 1 to a channel, but does not wait for the processing to finish. Does not have any output (except if an error occurs).
		\item [cawait] Monitors a channel and waits until specified conditions for the channel match. Then the value is displayed. This tool shall honor the default output format.
	\end{description}
	\item caTools shall be written in C programming language.
	\item caTools shall compile on the following platforms:
	\begin{itemize}
		\item SL6-x86 SL6-x86\_64,
		\item eldk52-e500v2,
		\item eldk42-ppc4xxFP.
	\end{itemize}
	\item Input and expected output shall be backwards compatible with existing PSI tools, except for \textbf{caInfo} tool and array handling.
	\item The user shall be able to request specified compound data type (eg. \texttt{DBR\_TIME\_CHAR}).
	\item The user shall be able to select channel access timeout (eg. how long to wait for channel to connect).
	\item Date and time of the record execution shall be available (CA server date/time).
	\item Date and time of the record printout shall be available (client local date/time).
\end{enumerate}

\subsection{Tool specifics}
\begin{enumerate}
	\item \textbf{camon} shall have options to select incremental time-stamp:
	\begin{itemize}
		\item Time elapsed since start of program.
		\item Time elapsed since last update.
		\item Time elapsed since last update, by channel.
	\end{itemize}
	\item \textbf{camon} shall have an option to exit monitoring after user specified number of updates.
	\item \textbf{cawait} shall have an option to exit after a user specified timeout.
	\item \textbf{caInfo} tool shall display all the information available in \texttt{DBR\_CTRL\_XXX} compound data type.
\end{enumerate}

\subsection{Value formatting}
\begin{enumerate}
	\item The format of the \texttt{decimal} values shall be settable:
	\begin{itemize}
		\item Round the value to closest integer, round to next integer, round to previous integer.
		\item Scientific notation (mantissa/exponent) with specified precision - \%e format.
		\item Decimal floating point notation with specified precision - \%f format.
		\item Use the shortest representation between \%e and \%f.
		\item Override the PREC field defined in the record.
	\end{itemize}
	\item The user shall be able to set the format of the \texttt{integer} values to hex, binary or octal.
	\item The user shall be able to force interpretation of \texttt{enum} and \texttt{char} values as integers or characters / array of characters.
\end{enumerate}

\subsection{Array handling}
\begin{enumerate}
	\item Arrays shall be printed as a list of values, where the formatting shall be based on the value type (eg. char type is displayed as an ASCII character, double type is displayed as a decimal number).
	\item Array values separator shall be configurable.
	\item The user shall be able to set the requested amount of array elements to read or write.
%	\item The user shall be able to display the number of elements in an array (NORD record field) before printing out array values.
\end{enumerate}

\subsection{Output formatting}
\begin{enumerate}
	\item Severity and status shall be displayed when not in \texttt{NO\_ALARM} state.
	\item The user shall be able for force the severity and status to always be displayed.
	\item The user shall be able for force the severity and status to never be displayed.
	\item The user shall be able to hide channel name from the output.
	\item The user shall be able to hide engineering units (EGU) from displaying next to values.
	\item The value and output formatting shall be applied to all tools that honor the default output format equally.
\end{enumerate}

\bibliographystyle{plain}
\bibliography{requirementReferences}

\end{document}
